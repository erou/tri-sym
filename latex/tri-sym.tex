\documentclass[a4paper,11pt]{article}

\usepackage[a4paper,left=2.5cm,right=2.5cm,top=3.5cm,bottom=3.5cm]{geometry}
\usepackage[utf8]{inputenc}
\usepackage[T1]{fontenc}
\usepackage[english]{babel}
\usepackage{graphicx}
\usepackage{amsmath,amssymb,amsthm,amsopn}
\usepackage{mathrsfs}
\usepackage{graphicx}
\usepackage{array}
\usepackage{makecell}
\usepackage{bm}
\usepackage{hyperref}
\usepackage[shortlabels]{enumitem}
\hypersetup{
    colorlinks=true,
    linkcolor=blue,
    citecolor=red,
}
\usepackage{diagbox}

\usepackage{algorithm}
\usepackage{algpseudocode}

\renewcommand{\algorithmicrequire}{\textbf{Input:}}
\renewcommand{\algorithmicensure}{\textbf{Output:}}

%\usepackage[top=1cm,bottom=1cm]{geometry}
%\usepackage{listings}
%\usepackage{xcolor}

\usepackage{tikz}
\usetikzlibrary{arrows}

% Tikz style

\tikzset{round/.style={circle, draw=black, very thick, scale = 0.7}}
\tikzset{arrow/.style={->, >=latex}}
\tikzset{dashed-arrow/.style={->, >=latex, dashed}}

\newtheoremstyle{break}%
{}{}%
{\itshape}{}%
{\bfseries}{}%  % Note that final punctuation is omitted.
{\newline}{}

\newtheoremstyle{sc}%
{}{}%
{}{}%
{\scshape}{}%  % Note that final punctuation is omitted.
{\newline}{}

\theoremstyle{break}
\newtheorem{thm}{Theorem}[section]
\newtheorem{lm}[thm]{Lemma}
\newtheorem{prop}[thm]{Proposition}
\newtheorem{cor}[thm]{Corollary}

\theoremstyle{sc}
\newtheorem{exo}{Exercice}

\theoremstyle{definition}
\newtheorem{defi}[thm]{Definition}
\newtheorem{ex}[thm]{Example}

\theoremstyle{remark}
\newtheorem{rem}[thm]{Remark}

% Math Operators

\DeclareMathOperator{\Card}{Card}
\DeclareMathOperator{\Gal}{Gal}
\DeclareMathOperator{\Id}{Id}
\DeclareMathOperator{\Img}{Im}
\DeclareMathOperator{\Ker}{Ker}
\DeclareMathOperator{\Minpoly}{Minpoly}
\DeclareMathOperator{\Mod}{mod}
\DeclareMathOperator{\Ord}{Ord}
\DeclareMathOperator{\ppcm}{ppcm}
\DeclareMathOperator{\tr}{Tr}
\DeclareMathOperator{\Vect}{Vect}
\DeclareMathOperator{\Span}{Span}
\DeclareMathOperator{\rank}{rank}
\DeclareMathOperator{\ev}{ev}

% Shortcuts

\newcommand{\dE}{\partial(E)}
\newcommand{\dF}{\partial(F)}
\newcommand{\dG}{\partial(G)}
\newcommand{\diff}{\mathop{}\!\mathrm{d}}
\newcommand{\eg}{\emph{e.g. }}
\newcommand{\emb}{\hookrightarrow}
\newcommand{\embed}[2]{\phi_{#1\hookrightarrow#2}}
\newcommand{\ent}[2]{[\![#1,#2]\!]}
\newcommand{\ie}{\emph{i.e. }}
\newcommand{\ps}[2]{\left\langle#1,#2\right\rangle}
\newcommand{\eqdef}{\overset{\text{def}}{=}}
\newcommand{\f}{f}%{\mathfrak{f}}
\newcommand{\bff}{\mathbf{f}}
\newcommand{\E}{\mathcal{E}}
\newcommand{\A}{\mathcal{A}}
\newcommand{\B}{\mathcal{B}}
\newcommand{\R}{\mathcal{R}}
\newcommand{\bfa}{\mathbf{a}}
\newcommand{\bfb}{\mathbf{b}}
\newcommand{\D}{\mathcal{D}}
\newcommand{\Pcal}{\mathcal{P}}
\newcommand{\musym}{\mu^{\textrm{sym}}}
\newcommand{\mutri}{\mu^{\textrm{tri}}}
\newcommand{\muhyp}{\mu^{\textrm{hyp}}}
\newcommand{\musymG}[1][G]{\mu^{\textrm{sym},#1}}
\newcommand{\mutriG}[1][G]{\mu^{\textrm{tri},#1}}
\newcommand{\muhypG}[1][G]{\mu^{\textrm{hyp},#1}}
\newcommand{\hmusym}{\hat\mu^{\textrm{sym}}}
\newcommand{\hmutri}{\hat\mu^{\textrm{tri}}}
\newcommand{\hmuhyp}{\hat\mu^{\textrm{hyp}}}
\newcommand{\hmusymG}[1][G]{\hat\mu^{\textrm{sym},#1}}
\newcommand{\hmutriG}[1][G]{\hat\mu^{\textrm{tri},#1}}
\newcommand{\hmuhypG}[1][G]{\hat\mu^{\textrm{hyp},#1}}
\newcommand{\Msym}{M^{\textrm{sym}}}
\newcommand{\Mtri}{M^{\textrm{tri}}}
\newcommand{\Mhyp}{M^{\textrm{hyp}}}
\newcommand{\hMsym}{\hat{M}^{\textrm{sym}}}
\newcommand{\hMtri}{\hat{M}^{\textrm{tri}}}
\newcommand{\hMhyp}{\hat{M}^{\textrm{hyp}}}
\newcommand{\tri}[2]{\mu_{#1}^{\text{tri}}(#2)}
\newcommand{\sym}[2]{\mu_{#1}^{m_3}(#2)}
\newcommand{\K}{\mathbf{k}}


% opening
\title{Tri-symmetric decompositions of bilinear maps over finite fields}
\author{}



\begin{document}

\maketitle

\begin{abstract}
  To be written.
\end{abstract}

%\tableofcontents

%\clearpage

\section{Introduction}
\label{sec:intro}

\paragraph{Bilinear complexity.} Let $k=\mathbb{F}_{q}$ be a finite field with
$q$ elements and $\A$
an algebra over $k$. Given an algorithm dealing with elements in $\A$, one is
usually interested in the (asymptotic) cost of the algorithm. In order to
understand this cost, one studies the \emph{complexity} of the algorithm, \ie
the number of operations needed by the algorithm. We can for example count the number
of bit operations or the algebraic operations $(+, \times, \cdot)$ in the algebra
$\A$ or in the field $k$. The latter is called the \emph{algebraic complexity}
and in this model we suppose that all algebraic operations have the same cost.
Nevertheless, multiplications in an algebra $\A$ are arguably more expensive than
additions or scalar multiplications. In the context of the computation of
bilinear maps, extensive work has been done to reduce the number of
multiplications involved. Notable examples are Karatsuba's
algorithm~\cite{Karatsuba63} and
Strassen's algorithm~\cite{Strassen69}. Karatsuba's algorithm is
based on the fact that the bilinear map associated to the product of two
polynomials of degree $1$
\[
  A = a_1 X + a_0\text{ and }B = b_1 X + b_0
\]
can be computed with three products $a_0b_0, (a_0+a_1)(b_0+b_1), a_1b_1$ instead
of the four classic ones $a_0b_0, a_0b_1, a_1b_0, a_1b_1$. Strassen's algorithm
exploits a similar idea in the case of $2\times2$ matrices: only $7$ products
are used instead of $8$ in order to compute a matrix product. Both these
algorithms have very practical consequences. The \emph{bilinear complexity}
$\mu(\A/k)$ of the algebra $\A$ represents the minimum number of two-variable
multiplications in $k$ needed to compute a product in $\A$, assuming that other
operations such as addition or multiplication by a constant have no cost. The
bilinear complexity is defined as the minimal number $n$ such that there exist
linear forms $(\varphi_i)_{1\leq i \leq n}$, $(\psi_i)_{1\leq i \leq n}$ in
$\A^\vee$, where $\A^\vee$ is the dual of $\A$, and
elements $(\alpha_i)_{1\leq i \leq n}$ in $\A$ such that
\[
  \forall x, y\in\A,\,xy = \sum_{i=1}^{n}\varphi_i(x)\psi_i(y)\alpha_i.
\]
Equivalently, it can be defined as the rank of the tensor in
\[
  \A^\vee \times \A^\vee\times \A
\]
corresponding to the multiplication map in $\A$. Let $k^{2\times2}$ be the space
of $2\times2$ matrices over $k$. We know thanks to Strassen's algorithm that 
\[
  \mu(k^{2\times 2}/k) \leq 7.
\]
In fact, this is optimal, so we have exactly $\mu(k^{2\times2}/k)=7$. In
general, it seems to be hard to find the bilinear complexity of a given algebra,
for example the bilinear complexity of $k^{3\times3}$ is not known.
In the litterature, work has been done both to algorithmically find the bilinear complexity of
small algebras~\cite{BDEZ12, Covanov18} and to understand how the bilinear
complexity asymptotically grows~\cite{CC88, BCPRRR}. Chudnovsky and Chudnosky
proved in 1988 that the bilinear complexity of an extension field
$\mathbb{F}_{q^n}/\mathbb{F}_{q}$ was linear in the the degree $n$ of the
extension, using an evaluation-interpolation method on curves. In this article, we
investigate both questions, in the case of an other type of decomposition that
is called \emph{tri-symmetric}. We define symmetric and tri-symmetric
decomposition in Section~\ref{sec:symtrisym}. In Section~\ref{sec:algos} we
describe algorithms to compute tri-symmetric decompositions. Finally, in
Section~\ref{sec:asymptotic}, we prove that the tri-symmetric bilinear
complexity of an extension of a finite field is again linear in the degree of the
extension.

\section{Symmetric and tri-symmetric decompositions}
\label{sec:symtrisym}

When the algebra $\A$ is commutative, \ie the product is symmetric, it is
natural to search for \emph{symmetric decompositions}, \ie formulas of the form
\begin{equation}
  \forall x, y\in\A,\,xy = \sum_{i=1}^{n}\varphi_i(x)\varphi_i(y)\alpha_i,
  \label{eq:sym}
\end{equation}
where for all $1\leq i\leq n$, $\varphi_i\in\A^\vee$ and $\alpha_i\in\A$. On an
algorithmic point of view, it is also simpler to find symmetric decompositions,
because the search space is smaller. The minimal $n$ such that we have an
equation of type~\eqref{eq:sym} is called the \emph{symmetric bilinear
complexity} and we write $\mu_{\text{sym}}(\A/k)$. If $\A$ is commutative, then
we always have $\mu_{\text{sym}}(\A/k)<\infty$, and it is clear that 
\[
  \mu(\A/k)\leq\mu_{\text{sym}}(\A/k),
\]
but we do not know an example where the two quantities are different.
It was also shown\footnote{Where?} that the
asymptotic \emph{symmetric} bilinear complexity of a finite extension of a
finite field is linear in the degree of the extension.

In the case $\A=\mathbb{F}_{q^n}$ of a finite field extension, every linear form
$\varphi\in\A^\vee$ can be written as $x\mapsto\tr(x\alpha)$, where
$\alpha\in\A$ and $\tr$ is the trace of the extension
$\mathbb{F}_{q^n}/\mathbb{F}_q$. Letting $\ps{x}{\alpha}=\tr(x\alpha)$, we can
rewrite~\eqref{eq:sym} as
\begin{equation}
  \forall x, y\in\A,\,xy = \sum_{i=1}^{n}\ps{x}{\beta_i}\ps{y}{\beta_i}\alpha_i,
  \label{eq:sym-tr}
\end{equation}
with $\alpha_i, \beta_i\in\A$ for all $1\leq i\leq n$. An other question,
already invastigated in~\cite{SL81}, is then
whether we can find decompositions with even more symmetries. We say that we
have a \emph{tri-symmetric} decomposition if there exist elements $(\alpha_i)_{1\leq i
\leq n}$ in $\A$ and scalars $(\lambda_i)_{1\leq i \leq n}$ in $k$ such that
\begin{equation}
  \forall x, y\in\A,\,xy =
  \sum_{i=1}^{n}\lambda_i\ps{x}{\alpha_i}\ps{y}{\alpha_i}\alpha_i.
  \label{eq:tri-sym}
\end{equation}
Such a decomposition always exist if $q\neq2$. As before, the minimal $n$
in~\eqref{eq:tri-sym} is called the \emph{tri-symmetric bilinear complexity} and
is noted $\mu_{\text{tri}}(\A/k)$. Once again, it is clear that
\[
  \mu(\A/k) \leq \mu_{\text{sym}}(\A/k)\leq\mu_{\text{tri}}(\A/k),
\]
but, for $q\neq2$, we do not know any example where these quantities are
different.

\section{Finding tri-symmetric decompositions}
\label{sec:algos}

\section{Asymptotic lengths}
\label{sec:asymptotic}

\end{document}
