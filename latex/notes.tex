\documentclass[a4paper,11pt]{article}

\usepackage[a4paper,left=2.5cm,right=2.5cm,top=3.5cm,bottom=3.5cm]{geometry}
\usepackage[utf8]{inputenc}
\usepackage[T1]{fontenc}
\usepackage[english]{babel}
\usepackage{graphicx}
\usepackage{amsmath,amssymb,amsthm,amsopn}
\usepackage{mathrsfs}
\usepackage{graphicx}
\usepackage{array}
\usepackage{makecell}
\usepackage{bm}
\usepackage{hyperref}
\usepackage[shortlabels]{enumitem}
\hypersetup{
    colorlinks=true,
    linkcolor=blue,
    citecolor=red,
}
\usepackage{diagbox}

\usepackage{algorithm}
\usepackage{algpseudocode}

\renewcommand{\algorithmicrequire}{\textbf{Input:}}
\renewcommand{\algorithmicensure}{\textbf{Output:}}

%\usepackage[top=1cm,bottom=1cm]{geometry}
%\usepackage{listings}
%\usepackage{xcolor}

\usepackage{tikz}
\usetikzlibrary{arrows}

% Tikz style

\tikzset{round/.style={circle, draw=black, very thick, scale = 0.7}}
\tikzset{arrow/.style={->, >=latex}}
\tikzset{dashed-arrow/.style={->, >=latex, dashed}}

\newtheoremstyle{break}%
{}{}%
{\itshape}{}%
{\bfseries}{}%  % Note that final punctuation is omitted.
{\newline}{}

\newtheoremstyle{sc}%
{}{}%
{}{}%
{\scshape}{}%  % Note that final punctuation is omitted.
{\newline}{}

\theoremstyle{break}
\newtheorem{thm}{Theorem}[section]
\newtheorem{lm}[thm]{Lemma}
\newtheorem{prop}[thm]{Proposition}
\newtheorem{cor}[thm]{Corollary}

\theoremstyle{sc}
\newtheorem{exo}{Exercice}

\theoremstyle{definition}
\newtheorem{defi}[thm]{Definition}
\newtheorem{ex}[thm]{Example}

\theoremstyle{remark}
\newtheorem{rem}[thm]{Remark}

% Math Operators

\DeclareMathOperator{\Card}{Card}
\DeclareMathOperator{\Gal}{Gal}
\DeclareMathOperator{\Id}{Id}
\DeclareMathOperator{\Img}{Im}
\DeclareMathOperator{\Ker}{Ker}
\DeclareMathOperator{\Minpoly}{Minpoly}
\DeclareMathOperator{\Mod}{mod}
\DeclareMathOperator{\Ord}{Ord}
\DeclareMathOperator{\ppcm}{ppcm}
\DeclareMathOperator{\tr}{Tr}
\DeclareMathOperator{\Vect}{Vect}
\DeclareMathOperator{\Span}{Span}
\DeclareMathOperator{\rank}{rank}
\DeclareMathOperator{\ev}{ev}

% Shortcuts

\newcommand{\dE}{\partial(E)}
\newcommand{\dF}{\partial(F)}
\newcommand{\dG}{\partial(G)}
\newcommand{\diff}{\mathop{}\!\mathrm{d}}
\newcommand{\eg}{\emph{e.g. }}
\newcommand{\emb}{\hookrightarrow}
\newcommand{\embed}[2]{\phi_{#1\hookrightarrow#2}}
\newcommand{\ent}[2]{[\![#1,#2]\!]}
\newcommand{\ie}{\emph{i.e. }}
\newcommand{\ps}[2]{\left\langle#1,#2\right\rangle}
\newcommand{\eqdef}{\overset{\text{def}}{=}}
\newcommand{\f}{f}%{\mathfrak{f}}
\newcommand{\bff}{\mathbf{f}}
\newcommand{\E}{\mathcal{E}}
\newcommand{\A}{\mathcal{A}}
\newcommand{\B}{\mathcal{B}}
\newcommand{\R}{\mathcal{R}}
\newcommand{\bfa}{\mathbf{a}}
\newcommand{\bfb}{\mathbf{b}}
\newcommand{\D}{\mathcal{D}}
\newcommand{\Pcal}{\mathcal{P}}
\newcommand{\musym}{\mu^{\textrm{sym}}}
\newcommand{\mutri}{\mu^{\textrm{tri}}}
\newcommand{\muhyp}{\mu^{\textrm{hyp}}}
\newcommand{\musymG}[1][G]{\mu^{\textrm{sym},#1}}
\newcommand{\mutriG}[1][G]{\mu^{\textrm{tri},#1}}
\newcommand{\muhypG}[1][G]{\mu^{\textrm{hyp},#1}}
\newcommand{\hmusym}{\hat\mu^{\textrm{sym}}}
\newcommand{\hmutri}{\hat\mu^{\textrm{tri}}}
\newcommand{\hmuhyp}{\hat\mu^{\textrm{hyp}}}
\newcommand{\hmusymG}[1][G]{\hat\mu^{\textrm{sym},#1}}
\newcommand{\hmutriG}[1][G]{\hat\mu^{\textrm{tri},#1}}
\newcommand{\hmuhypG}[1][G]{\hat\mu^{\textrm{hyp},#1}}
\newcommand{\Msym}{M^{\textrm{sym}}}
\newcommand{\Mtri}{M^{\textrm{tri}}}
\newcommand{\Mhyp}{M^{\textrm{hyp}}}
\newcommand{\hMsym}{\hat{M}^{\textrm{sym}}}
\newcommand{\hMtri}{\hat{M}^{\textrm{tri}}}
\newcommand{\hMhyp}{\hat{M}^{\textrm{hyp}}}
\newcommand{\tri}[2]{\mu_{#1}^{\text{tri}}(#2)}
\newcommand{\sym}[2]{\mu_{#1}^{m_3}(#2)}
\newcommand{\K}{\mathbf{k}}


% opening
\title{Notes on tri-symmetric formulas}
\author{}



\begin{document}

\maketitle

\begin{abstract}
These notes aim at describing some algorithms to search for tri-symmetric
decompositions of bilinear maps in finite fields. They do not provide a thorough
presentation of the background nor the interest that such decompositions have.
In order to have some of these elements, the reader may want to look
at~\cite{BDEZ12, Covanov18, Covanov19}.
\end{abstract}

%\tableofcontents

%\clearpage

\section{Introduction}
\label{intro}

Let $p\in\mathbb{N}$ be a prime number, $n\in\mathbb{N}$ an integer and
$\mathbb{F}_{p^n}/\mathbb{F}_{p}$ a finite field extension of
$\mathbb{F}_{p}=\mathbb{Z}/p\mathbb{Z}$. Let $\mathcal{M}$ be the set of
bilinear \emph{maps} from $\mathbb{F}_{p^n}\times\mathbb{F}_{p^n}$ to
$\mathbb{F}_{p^n}$ and let $\mathcal F$ the set of bilinear \emph{forms} from $\mathbb{F}_{p^n}\times\mathbb{F}_{p^n}$ to
$\mathbb{F}_{p}$. Let $\Phi\in\mathcal M$ a bilinear map, our goal is to find an
equation of the form
\begin{equation}
  \Phi(x, y) = \sum_{j=1}^r f_j(x, y)\alpha_j
  \label{1}
\end{equation}
with $f_j\in\mathcal F$ bilinear forms of rank $1$ and $\alpha_j\in\mathbb{F}_{p^n}$ for all $1\leq
j \leq r$. If the integer $r$ in Equation~\eqref{1} is minimal, then we say that
$r$ is the \emph{bilinear rank} of $\Phi$. Being a rank $1$ bilinear form is also equivalent to be the product
of two linear forms, so Equation~\eqref{1} can also be written
\begin{equation}
  \Phi(x, y) = \sum_{j=1}^r g_j(x)h_j(y)\alpha_j.
  \label{2}
\end{equation}

If $\Phi$ is a symmetric map, we can also try to search for symmetric formulas,
\ie formulas where the $f_j$ are symmetric, or in other words where $g_j = h_j$
for all $j$:

\begin{equation}
  \Phi(x, y) = \sum_{j=1}^r g_j(x)g_j(y)\alpha_j.\footnote{Note that $r$ is not
    necessarily the same as in Equation~\eqref{1} and~\eqref{2}. It is not known
  if the minimal $r$ such that this kind of equation holds, that we could maybe
  call \emph{symmetric rank}, is the same as the bilinear rank.}
  \label{sym}
\end{equation}

Furthermore, any linear form $g:\mathbb{F}_{p^n}\to\mathbb{F}_p$ can be written
$g(x) = \tr(\alpha x)$ with $\alpha\in\mathbb{F}_{p^n}$ a given element
depending on $g$ and $\tr$ the trace of $\mathbb{F}_{p^n}$ over
$\mathbb{F}_p$. Thus, we can rewrite Equation~\eqref{sym} as
\begin{equation}
  \Phi(x, y) = \sum_{j=1}^r \tr(\beta_j x)\tr(\beta_j y)\alpha_j.
  \label{symtr}
\end{equation}
Finally we may want to add yet another layer of symmetry, we can ask
$\beta_j=\alpha_j$ for all $j$, and we will call these last decompositions
\emph{tri-symmetric}:
\begin{equation}
  \Phi(x, y) = \sum_{j=1}^r \lambda_j\tr(\alpha_j x)\tr(\alpha_j
  y)\alpha_j\footnote{Once again, when such a decomposition exists, it is not known if the \emph{tri-symmetric
  rank} is stricly larger than the other ranks.}
  \label{trisym}
\end{equation}
where $\lambda_j\in\mathbb{F}_p$ for all $1\leq j \leq r$.

\section{From bilinear maps to vector space~\cite[Ch. 1]{Covanov18}}

We keep the same noations as in Section~\ref{intro}. We can choose a basis
of $\mathbb{F}_{p^n}/\mathbb{F}_p$ and write
\[
  \Phi = (\varphi_1, \dots, \varphi_n)
\]
with bilinear forms $\varphi_i\in\mathcal F$ for all $1\leq i \leq n$. The set
$\mathcal F$ of bilinear forms is a $\mathbb{F}_{p}$-vector space of dimension
$n^2$. We then consider the subspace 
\[
  V=\Span(\varphi_1, \dots, \varphi_n)
\]
and the sets
\[
  \mathcal G = \left\{ g\in\mathcal F\,|\,g\text{ is of rank }1 \right\}
\]
and
\[
  \mathcal G_{\text{sym}} = \left\{ g\in\mathcal F\,|\,g\text{ is symmetric and of rank }1 \right\}.
\]
A strategy to find formulas is to search for elements $g_1, \dots,
g_r\in\mathcal G$ that span $V$. It is similar to an exhaustive search through
$\mathcal G$, but Covanov~\cite{Covanov19} developed clever ideas to exploit the
symmetries of the bilinear map $\Phi$ and substantially reduce the search space.
Of course, searching for symmetric formulas implies searching in $\mathcal
G_{\text{sym}}$, which is smaller than $\mathcal G$
\[
  \Card \mathcal G_{\text{sym}} = \sqrt{\Card\mathcal G},
\]
so searching for symmetric formulas is faster.

\section{The tri-symmetric world}

We cannot apply Covanov's ideas to the tri-symmetric case because the strategy
used does not take into account the fact that, in the tri-symmetric case, each
symmetric bilinear form comes wih an element in $\mathbb{F}_{p^n}$ that
determines its
contribution to $\Phi=(\varphi_1, \dots, \varphi_n)\in\mathcal B$. 

A first idea to compute a tri-symmetric decomposition of length $r$ is to
exhaustively search through $r$-uples $(\alpha_1, \dots, \alpha_r)$ and
$(\lambda_1, \dots, \lambda_r)$ to see if we have
\[
  \Phi(x, y) = \sum_{j=1}^r \lambda_j\tr(\alpha_j x)\tr(\alpha_j y)\alpha_j.
\]

Of course, it has a quite horrific complexity, so we present other solutions.
Nonetheless, all our solutions have an exponential complexity. 
It will be convenient to add some more notations.
Let $\f$ be the application mapping an element in
$\mathbb{F}_{p^n}$ to its associated bilinear symmetric form:
\[
\begin{array}{cccc}
  \f: & \mathbb{F}_{p^n} & \to & \mathcal F \\
  & \alpha & \mapsto & (x, y)\mapsto \tr(\alpha x)\tr(\alpha y).
\end{array}
\]
For some choice\footnote{We will see that our algorithms do not depend on this
choice.} of a basis of $\mathbb{F}_{p^n}$ over $\mathbb{F}_p$, for all
$1\leq i\leq n$ we let
\[
  \E_i=\left\{ \alpha=(a_1, \dots, a_n)\in\mathbb{F}_{p^n}
  \,|\, \forall k\leq i-1,\,a_k=0\text{ and }a_i=1 \right\}
\]
and
\[
  \E = \bigcup_{i=1}^n\E_i.
\]
Since we are allowed to use scalar coefficients $\lambda_j\in\mathbb{F}_p$ in our decomposition,
we can search for elements $\alpha_j\in\mathbb{F}_{p^n}$ that are in fact in
$\E$. The idea is always the same and is rather simple: instead of trying to find a
\emph{global} solution directly, we try to find \emph{local}\footnote{Nothing to
do with the local solutions of differential equations.} solutions, \ie coordinates by
coordinates.

\paragraph{The vector space strategy.}
In order to search for a length $r$ tri-symmetric decomposition, we
search for a decomposition of length $s\leq r$\footnote{If $s=r$, it is the same
as a global exhaustive search.} on each coordinate: we start by exhaustively
search for up to $s$ elements in $\E_1$ such that
\[
  \varphi_1\in\Span(\f(\alpha_1), \dots, \f(\alpha_s)).
\]
We then obtain scalars $\lambda_1, \dots, \lambda_s$ such that
\[
  \Phi-\sum_{j=1}^s\lambda_j\f(\alpha_j)\alpha_j=(0, \varphi_2', \dots,
  \varphi_n')
\]
and we go on and search for up to $s$ elements in $\E_2$ in order to eliminate
$\varphi_2'$. Since the elements are
found in $\E_2$, they have no contribution on the first coordinate. We then
continue this process with $\varphi_3''$, and so on until we find a
tri-symmetric decomposition. It is
possible that there exist tri-symmetric decompositions with $r$ elements that
are all in $\E_1$, so if our parameter $s$ is strictly less than $r$, we may not
have all the decompositions. This strategy is described in
Algorithm~\ref{algo:vect}.

\begin{algorithm}
  \caption{(Vector space strategy)}\label{algo:vect}
  \begin{algorithmic}[1]
    \Require{$\Phi=(\varphi_1, \dots, \varphi_n)\in\mathcal B$ a bilinear map;
      $r, s\in\mathbb{N}$ two integers with $s\leq r$}
    \Ensure{A list of tri-symmetric decompositions of $\Phi$ with up to $r$ elements.}

    \Procedure{FindDecomposition}{$\varphi, E, t, R_{\text{glob}},
    R_{\text{loc}}=\emptyset$}
    \If{$\varphi\in\Span(R_\text{loc})$}
    \State $C\gets \left\{ (\lambda_1, \alpha_1), \dots, (\lambda_u, \alpha_u)
    \right\}$, where $\varphi=\sum_{j=1}^u \lambda_j\f(\alpha_j)$ and
    $R_\text{loc}=\left\{ \alpha_1, \dots, \alpha_u \right\}$
    \State $R_\text{glob}\gets R_\text{glob}\bigcup\left\{C\right\}$
    \ElsIf{$\Card R_\text{loc}<t$}
    \For{$i=1$ to $\Card E$}
    \State $E'\gets \left\{ e_{i+1}, \dots, e_v \right\}$\Comment{$E=\left\{ e_1,
    \dots, e_v\right\}$}
    \State \Call{FindDecomposition}{$\varphi, E', t, R_\text{glob},
    R_\text{loc}\cup\left\{ e_i \right\}$}
    \EndFor
    \EndIf
    \EndProcedure

    \Procedure{TriSymmetricSearch}{$\Phi, r, s, S_\text{glob}, j=1,
    S_\text{loc}=\emptyset$}
      \If{$\Phi = 0$}
      \State $S_\text{glob}\gets S_\text{glob}\bigcup \left\{S_\text{loc}\right\}$. 
      \Else
        \State $t\gets\min(s, r)$
        \State $\mathcal R\gets\emptyset$
        \State \Call{FindDecomposition}{$\varphi_j, \E_j, t,
        \mathcal R$}\Comment{$\Phi=(\varphi_1, \dots, \varphi_n)$}
        \ForAll{$S\in\mathcal R$}
        \State $\Phi'\gets\Phi-\sum_{(\lambda, \alpha)\in
        S}\lambda\f(\alpha)\alpha$
        \State \Call{TriSymmetricSearch}{$\Phi', r-\Card S, s, S_\text{glob}, j+1,
          S_\text{loc}\cup S$}
        \EndFor
      \EndIf
    \EndProcedure
    \State $\mathcal S\gets\emptyset$
    \State \Call{TriSymmetricSearch}{$\Phi, r, s, \mathcal S$}
    \State \Return $\mathcal S$
  \end{algorithmic}
\end{algorithm}

\paragraph{The matrix strategy.}
The idea will still be the same, with a little change in the representation of
the bilinear forms. Instead of considering the bilinear forms
$\varphi\in\mathcal F$ as vectors in the $\mathbb{F}_{p}$-vector
space $\mathcal F$ of dimension $n^2$, we will work with their $n\times n$ matrix
representation. The main advantage of the matrix representation is that instead
of searching directly for all the $s$-uple $(\alpha_1, \dots, \alpha_s)$ such
that
\[
  \varphi\in\Span(\f(\alpha_1), \dots, \f(\alpha_s)),
\]
we will compute elements $\alpha_1, \dots, \alpha_s$ one by one by checking if
there exist $\lambda_1\in\mathbb{F}_p$ such that
\[
  \rank(\varphi-\lambda_1\f(\alpha_1))<\rank(\varphi),
\]
then $\lambda_2$ such that
\[
  \rank(\varphi-\lambda_1\f(\alpha_1)-\lambda_2\f(\alpha_2))<\rank(\varphi-\lambda_1\f(\alpha_1)),
\]
and so on, eliminating lots of other candidates on the way. This strategy
is described in Algorithm~\ref{algo:mat}.

\begin{algorithm}
  \caption{(Matrix strategy)}\label{algo:mat}
  \begin{algorithmic}[1]
    \Require{$\Phi=(\varphi_1, \dots, \varphi_n)\in\mathcal B$ a bilinear map;
      $r\in\mathbb{N}$ an integer}
    \Ensure{A list of tri-symmetric decompositions of $\Phi$ with up to $r$ elements.}

    \Procedure{FindDecMat}{$\varphi, E, R_{\text{glob}},
    R_{\text{loc}}=\emptyset$}
    \If{$\varphi=0$}\Comment{$\rank(\varphi)=0$}
    \State $R_\text{glob}\gets R_\text{glob}\bigcup\left\{R_\text{loc}\right\}$
    \Else
    \State $\mathcal C\gets\emptyset$
    \ForAll{$\alpha\in E$}
    \ForAll{$\lambda\in\mathbb{F}_p$}
    \If{$\rank(\varphi-\lambda\f(\alpha))<\rank(\varphi)$}
    \State $\mathcal C\gets\mathcal C\bigcup\left\{ (\lambda, \sigma) \right\}$
    \State \textbf{break}
    \EndIf
    \EndFor
    \EndFor
    \For{$i=1$ to $\Card\mathcal C$}\Comment{$\mathcal
      C=\left\{ (\lambda_1, \alpha_1), \dots, (\lambda_u, \alpha_u) \right\}$}
      \State $E'\gets\left\{ \alpha_{i+1}, \dots, \alpha_u \right\}$
    \State \Call{FindDecMat}{$\varphi-\lambda_i\f(\alpha_i), E', R_\text{glob},
    R_\text{loc}\cup\left\{ (\lambda_i, \alpha_i) \right\}$}
    \EndFor
    \EndIf
    \EndProcedure

    \Procedure{TriSymSearchMat}{$\Phi, r, S_\text{glob}, j=1,
    S_\text{loc}=\emptyset$}
      \If{$\Phi = 0$}
      \State $S_\text{glob}\gets S_\text{glob}\bigcup \left\{S_\text{loc}\right\}$. 
      \ElsIf{$\rank(\varphi_j)\leq r$}
        \State $\mathcal R\gets\emptyset$
        \State \Call{FindDecMat}{$\varphi_j, \E_j, 
        \mathcal R$}\Comment{$\Phi=(\varphi_1, \dots, \varphi_n)$}
        \ForAll{$S\in\mathcal R$}
        \State $\Phi'\gets\Phi-\sum_{(\lambda, \alpha)\in
        S}\lambda\f(\alpha)\alpha$
        \State \Call{TriSymSearchMat}{$\Phi', r-\Card S, S_\text{glob}, j+1,
          S_\text{loc}\cup S$}
        \EndFor
      \EndIf
    \EndProcedure
    \State $\mathcal S\gets\emptyset$
    \State \Call{TriSymmetricSearchMat}{$\Phi, r, \mathcal S$}
    \State \Return $\mathcal S$
  \end{algorithmic}
\end{algorithm}

The main problem of Algorithm~\ref{algo:mat} is that it only finds
\emph{minimal} solutions, in the sense that if $\varphi_1$ if of rank $r_1$,
\textsc{FindDecMat}$(\varphi_1, \mathcal E_1, \mathcal R)$ will find only
solutions composed of $r_1$ elements, not more. The same is also true for
$\varphi_2', \varphi_3''$, and so on. But there exist tri-symmetric
decompositions where there is more than $r_1$ elements that are in $\mathcal
E_1$ (as can be seen in Table~\ref{tab:sets}), so we would like a less naive version of Algorithm~\ref{algo:mat} that
allows for some margins on each coordinate. We denote this ``margin'' by
\[
  \mathfrak M = (m_1, \dots, m_n)\in \mathbb{N}^n
\]
and it means that for each step of the algorithm, if $\varphi_j$ is of rank
$r_j$, we are able to find solutions with up to $r_j + m_j$ elements. The new
method, namely Algorithm~\ref{algo:margin}, is then a generalization of
Algorithm~\ref{algo:mat} and Algorithm~\ref{algo:margin} called with $\mathfrak
M=(0, \dots, 0)$ is exactly Algorithm~\ref{algo:mat}.

\begin{algorithm}
  \caption{(Matrices with margins)}\label{algo:margin}
  \begin{algorithmic}[1]
    \Require{$\Phi=(\varphi_1, \dots, \varphi_n)\in\mathcal B$ a bilinear map;
      $r\in\mathbb{N}$ an integer, $\mathfrak M$ a margin}
    \Ensure{A list of tri-symmetric decompositions of $\Phi$ with up to $r$ elements.}

    \Procedure{FindDecMargin}{$\varphi, E, r, m, R_{\text{glob}},
    R_{\text{loc}}=\emptyset$}
    \If{$\rank\varphi\leq r$}
    \If{$\varphi=0$}\Comment{$\rank(\varphi)=0$}
    \State $R_\text{glob}\gets R_\text{glob}\bigcup\left\{R_\text{loc}\right\}$
    \ElsIf{$m=0$}\Comment{No margin left!}
    \State \Call{FindDecMat}{$\varphi, E, R_\text{glob},
    R_\text{loc}$}\Comment{"naive" strategy: Alg.~\ref{algo:mat}}
   \Else
   \For{$i=1$ to $\Card E$}\Comment{$E=\left\{ \alpha_1, \dots, \alpha_u \right\}$}
   \State $E'\gets\left\{ \alpha_{i+1}, \dots, \alpha_u \right\}$
    \ForAll{$\lambda\in\mathbb{F}_p$}
    \State $\varphi'\gets\varphi-\lambda\alpha_i$
    \State $R_\text{loc}'\gets R_\text{loc}\bigcup\left\{ (\lambda, \alpha_i)
    \right\}$
    \If{$\rank(\varphi')<\rank(\varphi)$}
    \State \Call{FindDecMargin}{$\varphi', E', r-1, m,
      R_\text{glob}, R_\text{loc}'$}
    \Else
    \State \Call{FindDecMargin}{$\varphi', E', r-1, m-1,
      R_\text{glob}, R_\text{loc}'$}
    \EndIf
    \EndFor
    \EndFor
    \EndIf
    \EndIf
    \EndProcedure

    \Procedure{TriSymSearchMargin}{$\Phi, r, \mathfrak M, S_\text{glob}, j=1,
    S_\text{loc}=\emptyset$}
      \If{$\Phi = 0$}
      \State $S_\text{glob}\gets S_\text{glob}\bigcup \left\{S_\text{loc}\right\}$. 
      \ElsIf{$\rank(\varphi_j)\leq r$}
      \State $\mathcal R\gets\emptyset$\Comment{$\mathfrak M = (m_1, \dots, m_n)$}
        \State \Call{FindDecMargin}{$\varphi_j, \E_j, r, m_j,
        \mathcal R$}\Comment{$\Phi=(\varphi_1, \dots, \varphi_n)$}
        \ForAll{$S\in\mathcal R$}
        \State $\Phi'\gets\Phi-\sum_{(\lambda, \alpha)\in
        S}\lambda\f(\alpha)\alpha$
        \State \Call{TriSymSearchMargin}{$\Phi', r\!-\!\Card S, \mathfrak M, S_\text{glob}, j+1,
          S_\text{loc}\cup S$}
        \EndFor
      \EndIf
    \EndProcedure
    \State $\mathcal S\gets\emptyset$
    \State \Call{TriSymSearchMargin}{$\Phi, r, \mathfrak M, \mathcal S$}
    \State \Return $\mathcal S$
  \end{algorithmic}
\end{algorithm}

\begin{ex}
 Let us see the case of the extension 
 \[
   \mathbb{F}_{3^3}\cong \mathbb{F}_{3}[x]/(x^3-x+1)\cong \mathbb{F}_3[z]
 \]
 over $\mathbb{F}_3$. There are $4$ tri-symmetric decomposition of length $6$ of
 the product:
 \begin{align*}
   d_1 &=\left\{(1, z^2 + z),
  (2, z^2 + z + 1),
  (1, z^2 - z + 1),
  (1, -z^2 + 1),
  (2, -z^2 + z + 1),
  (1, -z^2 -z + 1)\right\}\\
  d_2 &= \left\{(2, z^2),
  (2, z^2 + 1),
  (1, z^2 + z + 1),
  (2, -z^2 + 1),
  (1, -z^2 + z),
  (2, -z^2 - z + 1)\right\}\\
  d_3 &= \left\{(1, z),
  (1, z + 1),
  (2, -z + 1),
  (2, z^2 + z),
  (1, -z^2 + 1),
  (1, -z^2 + z)\right\}\\
  d_4 &= \left\{ (1, z^2),
  (1, z^2 + 1),
  (2, z^2 + z),
  (2, z^2 - z + 1),
  (2, -z^2 + z),
  (1, -z^2 + z + 1).\right\}
 \end{align*}

How the elements of each decomposition fall into the sets $\mathcal E$ is
described in Table~\ref{tab:sets}. Table~\ref{tab:algos} describes which
algorithm finds which decomposition, in this table $\mathcal A(s, r)$ denotes
Algorithm~\ref{algo:vect} with parameters $s$ and $r$, while $\mathcal B(m_1,
m_2, m_3)$ denotes Algorithm~\ref{algo:margin} with the margin $\mathcal
M=(m_1, m_2, m_3)$. Thus $\mathcal B(0,0,0)$ is Algorithm~\ref{algo:mat}. There
is a checkmark \checkmark when the algorithms succeed in finding the
decomposition.

 \begin{table}
   \centering
\begin{tabular}{|c||ccc|}
   \hline
   \diagbox{Decomposition}{Set} & $\mathcal E_1$ & $\mathcal E_2$
   & $\mathcal E_3$ \\
   \hline
   \hline
   $d_1$ & 5 & 1 & 0 \\
   $d_2$ & 4 & 1 & 1 \\
   $d_3$ & 3 & 3 & 0 \\
   $d_4$ & 3 & 2 & 1 \\
  \hline 
 \end{tabular}
 \caption{The elements of a decomposition and their belonging to a set $\mathcal
 E$.}
\label{tab:sets}
 \end{table}

 \begin{table}
   \centering
 \begin{tabular}{|c||cccc|}
   \hline
   \diagbox{\small{Decomposition}}{\small{Algorithm}} & $\mathcal A(3, 6)$ & $\mathcal A(6, 6)$
   & $\mathcal B(0, 0, 0)$ & $\mathcal B(2, 1, 0)$ \\
   \hline
   \hline
   $d_1$ & & \checkmark & & \checkmark \\
   $d_2$ & & \checkmark & & \checkmark \\
   $d_3$ & \checkmark & \checkmark & & \checkmark \\
   $d_4$ & \checkmark & \checkmark & \checkmark & \checkmark \\
  \hline 
 \end{tabular}

   \caption{The algorithms and the decompositions they find.}
   \label{tab:algos}
 \end{table}

\end{ex}

\section{Frobenius action}

Let $\sigma$ be the Frobenius automorphism of $\mathbb{F}_{p^n}/\mathbb{F}_p$.
Assume we have found a tri-symmetric decomposition of the product in
$\mathbb{F}_{p^n}$ with some elements $\alpha_1, \dots, \alpha_r\in
\mathbb{F}_{p^n}$ and
$\lambda_1, \dots, \lambda_r\in\mathbb{F}_p$, \ie for all
$x, y\in\mathbb{F}_{p^n}$, we have
\[
  xy = \sum_{j=1}^r \lambda_j \tr(\alpha_jx)\tr(\alpha_jy)\alpha_j.
\]
Then we have that
\[
  \sigma^{-1}(x)\sigma^{-1}(y) = \sum_{j=1}^r \lambda_j
  \tr(\alpha_j\sigma^{-1}(x))\tr(\alpha_j\sigma^{-1}(y))\alpha_j.
\]
Since for all $z\in\mathbb{F}_{p^n}$,
\[
  \tr(z) = \tr(\sigma(z)),
\]
we also have
\[
  \sigma^{-1}(x)\sigma^{-1}(y) = \sum_{j=1}^r \lambda_j
  \tr(\sigma(\alpha_j)x)\tr(\sigma(\alpha_j)y)\alpha_j.
\]
Finally, by applying $\sigma$ on the two sides of the equation, we obtain
\[
  xy = \sum_{j=1}^r \lambda_j \tr(\alpha_j^\sigma x)\tr(\alpha_j^\sigma y)\alpha_j^\sigma
\]
where $\alpha_j^\sigma \eqdef \sigma(\alpha_j)$. We see that the Frobenius
automorphism acts on the tri-symmetric decompositions. A first consequence is
that a single decomposition may give birth to other ones.

An other one is that we would like to be able to ``quotient'' our search by the
action of the Frobenius. We could do that with an exhaustive search but we
currently do not have a way to use this in the other algorithms. 


\section{Quadratic extensions}

Let $q=p^l$ denote a prime power, with $p\neq2$. In this section we investigate
the case of the quadratic extension 
\[
  \mathbb{F}_{q^2}/\mathbb{F}_q.
\]
It is always possible to find an element $\zeta\in\mathbb{F}_q$ such that 
\[
  \mathbb{F}_{q^2}\cong \mathbb{F}_{q}[x]/(x^2-\zeta)=\mathbb{F}_q(\alpha)
\]
where $\alpha=\bar x$ is the canonical generator of $\mathbb{F}_{q^2}$. With
this representation of $\mathbb{F}_{q^2}$, we have that
\[
  \tr(x_0 + x_1\alpha) = 2x_0,
\]
where $\tr$ is the trace of $\mathbb{F}_{q^2}$ over $\mathbb{F}_q$.
We represent
bilinear forms on $\mathbb{F}_{q^2}\times \mathbb{F}_{q^2}$ as $2\times2$
matrices, so that the product
\[
  ab = (a_0+a_1\alpha)(b_0+b_1\alpha)=a_0b_0+\zeta a_1b_1
  +(a_0b_1+a_1b_0)\alpha
\]
can be writen as
\[
  ab = 
  \begin{bmatrix}
    1 & 0 \\
    0 & \zeta
  \end{bmatrix}
  +
  \begin{bmatrix}
    0 & 1 \\
    1 & 0
  \end{bmatrix}\alpha.
\]
With these notations, the bilinear map 
\[
  (a, b)\mapsto \tr((x_0+x_1\alpha)a)\tr((x_0+x_1\alpha)b)
\]
is given by the matrix 
\[
  f(x_0+x_1\alpha)=
  \left(S 
  \begin{bmatrix}
    x_0 \\
    x_1
  \end{bmatrix}\right)
  \left(S 
  \begin{bmatrix}
    x_0 \\
    x_1
  \end{bmatrix}\right)^\intercal
\]
where
\[
  S =
  \begin{bmatrix}
    \tr(\alpha^{i+j})
  \end{bmatrix}_{0\leq i, j\leq 1}
  =
  \begin{bmatrix}
   2 & 0 \\
   0 & 2\zeta
  \end{bmatrix}.
\]
In other words we have
\[
  f(x_0+x_1\alpha) = 
  4\begin{bmatrix}
   x_0^2 & \zeta x_0x_1 \\
   \zeta x_0 x_1 & \zeta^2 x_1^2
  \end{bmatrix}.
\]
And we can finally check that
\[
  ab = (1-\zeta^{-1})4^{-1}f(1)\cdot1 +
  (8\zeta)^{-1}f(1+\alpha)\cdot(1+\alpha)+(8\zeta)^{-1}f(1-\alpha)(1-\alpha)
\]
so that the trisymmetric bilinear complexity for
$\mathbb{F}_{q^2}/\mathbb{F}_{q}$ is $3$.

\section{Other algebras of small degree}

Until now, we have studied $\mathbb{F}_q$ algebras that were of a special kind
because they were field extensions, and thus they were not only algebras but
also fields. More generally, we can study other $\mathbb{F}_q$-algebra $\A$ such
that 
\[
  \A \cong \A^\times,
\]
where $\A^\times$ is the dual of $\A$, and for which we know how to realize the
isomorphism. For the case
\[
  \A = \mathbb{F}_{q^n},
\]
the isomorphism is realized by the trace map.

An other interesting case is the algebra of truncated polynomials, \ie the algebra
\[
  \A = \mathbb{F}_{q}[x]/(x^n) = \mathbb{F}_{q}[\alpha],
\]
where $\alpha=\bar x$ is the canonical generator.
In this context, an element $\bfa\in\A$ can be written as
\[
  \bfa = \sum_{j=0}^{n-1}a_j \alpha^j
\]
and a map realizing the isomorphism $\A\cong\A^\times$ is given by
\[
  \R(\bfa)=\R\left(\sum_{j=0}^{n-1}a_j \alpha^j\right) = a_{n-1}.
\]

\paragraph{Algebras of degree $2$.} Let $q=p^l$ denote a prime power, with
$p\neq 2$, we first investigate the algebra
\[
  \A = \mathbb{F}_q[x]/(x^2) =\mathbb{F}_q[\alpha]
\]
where $\alpha=\bar x$ is the canonical generator of $\A$ and $\alpha^2=0$.
With this representation of $\A$, we have that
\[
  \R(x_0 + x_1\alpha) = x_1,
\]
where $\R$ is the map defined earlier.
We represent
bilinear forms on $\A\times\A$ as $2\times2$
matrices, so that the product
\[
  \bfa\bfb = (a_0+a_1\alpha)(b_0+b_1\alpha)=a_0b_0+(a_0b_1+a_1b_0)\alpha
\]
can be writen as
\[
  \bfa\bfb = 
  \begin{bmatrix}
    1 & 0 \\
    0 & 0
  \end{bmatrix}
  +
  \begin{bmatrix}
    0 & 1 \\
    1 & 0
  \end{bmatrix}\alpha.
\]
With these notations, the bilinear map 
\[
  (\bfa, \bfb)\mapsto \R((x_0+x_1\alpha)\bfa)\R((x_0+x_1\alpha)\bfb)
\]
is given by the matrix 
\[
  f(x_0+x_1\alpha)=
  \left(S 
  \begin{bmatrix}
    x_0 \\
    x_1
  \end{bmatrix}\right)
  \left(S 
  \begin{bmatrix}
    x_0 \\
    x_1
  \end{bmatrix}\right)^\intercal
\]
where
\[
  S =
  \begin{bmatrix}
    \R(\alpha^{i+j})
  \end{bmatrix}_{0\leq i, j\leq 1}
  =
  \begin{bmatrix}
   0 & 1 \\
   1 & 0
  \end{bmatrix}.
\]
In other words we have
\[
  f(x_0+x_1\alpha) = 
  \begin{bmatrix}
   x_1^2 & x_0x_1 \\
   x_0 x_1 & x_0^2
  \end{bmatrix}.
\]
And we can finally check that
\[
  \bfa\bfb = -f(1)\cdot1 +
  2^{-1}f(1+\alpha)\cdot(1+\alpha)+2^{-1}f(1-\alpha)\cdot(1-\alpha)
\]
so that the trisymmetric bilinear complexity for
$\A=\mathbb{F}_q[x]/(x^2)$ is $3$.

\paragraph{Algebras of degree $3$ and $4$.} Let $q=p^l$ denote a prime power,
with $p\notin\left\{ 2, 3 \right\}$. Here again, $\R$ and $f$ denote the maps defined
earlier (they are not the same in degree $3$ and degree $4$).

For $\bfa, \bfb\in\A=\mathbb{F}_{q}[x]/(x^3)$, we have
\begin{equation*}
  \begin{split}
  \bfa\bfb =&
  -f(1-\alpha-\alpha^2)\cdot(1-\alpha-\alpha^2)+3^{-1}f(\alpha+2\alpha^2)\cdot(\alpha+2\alpha^2)\\
  &+2^{-1}f(1-\alpha-2\alpha^2)\cdot(1-\alpha-2\alpha^2) -
  3^{-1}f(\alpha-\alpha^2)\cdot(\alpha-\alpha^2)\\
  &+2^{-1}f(1-\alpha)\cdot(1-\alpha).
  \end{split}
\end{equation*}
Thus the trisymmetric bilinear complexity for $\A=\mathbb{F}_q[x]/(x^3)$ is $5$.

For $\bfa, \bfb\in\A=\mathbb{F}_{q}[x]/(x^4)$, we have
\begin{equation*}
  \begin{split}
    \bfa\bfb =&\phantom{+}2^{-1}f(1-\alpha^2+\alpha^3)\cdot(1-\alpha^2+\alpha^3)-f(1-\alpha^2)\cdot(1-\alpha^2)\\
  &+12^{-1}f(\alpha+2\alpha^2+2\alpha^3)\cdot(\alpha+2\alpha^2+2\alpha^3)\\
  &-12^{-1}f(\alpha-2\alpha^2+2\alpha^3)\cdot(\alpha-2\alpha^2+2\alpha^3)\\
  &-6^{-1}f(\alpha+\alpha^2-\alpha^3)\cdot(\alpha+\alpha^2-\alpha^3)\\
  &+6^{-1}f(\alpha-\alpha^2-\alpha^3)\cdot(\alpha-\alpha^2-\alpha^3)\\
  &+2^{-1}f(1-\alpha^2-\alpha^3)\cdot(1-\alpha^2-\alpha^3).
  \end{split}
\end{equation*}
Thus the trisymmetric bilinear complexity for $\A=\mathbb{F}_q[x]/(x^4)$ is $7$.



\bibliographystyle{plain}
\bibliography{erou}

\end{document}
